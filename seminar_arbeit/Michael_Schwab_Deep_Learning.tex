\documentclass[conference, german]{IEEEtran}
\IEEEoverridecommandlockouts
% The preceding line is only needed to identify funding in the first footnote. If that is unneeded, please comment it out.
\usepackage{amsmath,amssymb,amsfonts}
\usepackage{algorithmic}
\usepackage{graphicx}
\usepackage{textcomp}
\usepackage{xcolor}
\usepackage[utf8]{inputenc}
\usepackage{acronym}
\usepackage{hyperref}
\usepackage[ngerman]{babel}
\usepackage[babel,german=quotes]{csquotes}
\usepackage[backend=biber, natbib=true]{biblatex}
\usepackage{tabularx}
\usepackage{array, multirow} % tabularx: auf Textbreite vergrößern
\usepackage{makecell}
\usepackage{booktabs}
\addbibresource{verzechnis.bib}

\begin{document}

\title{Deep Learning\\
{\footnotesize DVA-Seminar 2018}
}
\author{\IEEEauthorblockN{Michael Schwab}
\IEEEauthorblockA{\textit{Fakultät für Informatik} \\
\textit{Hochschule für angewandte Wissenschaften Augsburg}\\
Augsburg, Deutschland \\
michael.schwab@hs-augsburg.de}
}

\maketitle
\renewcommand{\abstractname}{Zusammenfassung}



\begin{abstract}
\end{abstract}

%\begin{IEEEkeywords}
%\end{IEEEkeywords}	
\section*{Lizens}
\href{http://creativecommons.org/licenses/by-nc/4.0/}{\includegraphics{img/by-nc.png}}\\
{Deep Learning} von {Michael Schwab} ist lizenziert unter einer
\href{http://creativecommons.org/licenses/by-nc/4.0/}{Creative Commons
	Namensnennung-Nicht kommerziell 4.0 International Lizenz}.
\section{Einleitung} 
\section{Konventionen}
\begin{acronym}
	\acro{knn}[KNN]{Künstliches Neuronales Netz}
	\acroplural{knn}[KNNs]{Künstliche Neuronale Netze}
	\acro{dl}[DL]{Deep Learning}
	\acro{ids}[IDS]{Iris-Datensatz}
	\acro{ml}[ML]{Machine Learning}
\end{acronym}
\section{Grundlagen}
\subsection{Datensätze}
Eine essentielle Kompenete des \ac{dl} sind Daten.
Ein Datensatz ist eine Sammlung von Daten in einem gleichen oder ähnlichen Format.
Möchte man \ac{dl} erlernen, steht man oft vor dem Problem keine passenden Daten parat zu haben.
Aus diesem Grund gibt es verschiedene bereits gut klassifizierte Datensätze im Internet zu finden, die den Einstieg in das \ac{dl} erleichtern. 
Ein Beispiel für einen dieser Datensätze ist der \ac{ids}\footnote{\url{https://archive.ics.uci.edu/ml/datasets/iris}}.
\\\\
Der \ac{ids} wurde 1936 von dem Statisker und Biologen Richard Fischer vorgestellt.
Der Datensatz besteht aus 150 Einträgen und umfasst drei verschiedene Gattungen der Iris Blume.
Jeder Eintrag einer einzelnen Blume wird jeweils durch sowohl durch die Blütenkelch Länge und Breite als auch durch die Blütenblatt Länge und Breite beschrieben.
Die drei verschiedenen Gattungen der Iris Blume heißen Iris setosa, Iris virginica und Iris versitosa \citep[vgl.][]{WIKI01}.
\begin{table}
	\caption{Einträge aus dem \ac{ids}}
	\label{table:ids}
	\centering
	\begin{tabular}{c c c c c c}
	\toprule
	{} &    Feature 1 & Feature 2 & Feature 3 & Feature 4 & Label \\
	\midrule
	0 &  5.1 &  3.5 &  1.4 &  0.2 &  Iris-setosa \\
	1 &  4.9 &  3.0 &  1.4 &  0.2 &  Iris-setosa \\
	2 &  4.7 &  3.2 &  1.3 &  0.2 &  Iris-setosa \\
	3 &  4.6 &  3.1 &  1.5 &  0.2 &  Iris-setosa \\
	4 &  5.0 &  3.6 &  1.4 &  0.2 &  Iris-setosa \\
	\midrule
	140 &  6.7 &  3.1 &  5.6 &  2.4 &  Iris-virginica \\
	141 &  6.9 &  3.1 &  5.1 &  2.3 &  Iris-virginica \\
	142 &  5.8 &  2.7 &  5.1 &  1.9 &  Iris-virginica \\
	143 &  6.8 &  3.2 &  5.9 &  2.3 &  Iris-virginica \\
	144 &  6.7 &  3.3 &  5.7 &  2.5 &  Iris-virginica \\
	\bottomrule
\end{tabular}
\end{table}
\\\\
Tabelle \ref{table:ids} zeigt einen Auszug aus dem \ac{ids}.
Anhand diesem Beispiels sollen drei Begriffe erklärt werden, die sehr häufig 
im Bereich des \ac{ml} und \ac{dl} vorkommen.

\paragraph{Sample}
Ein Sample ist ein komplettes Beispiel aus einem Datensatz. 
In den meißten Fällen ist ein Sample mit einer Reihe in einem Datensatz gleichzusetzen.
In dem Fall des \ac{ids} entspricht ein Sample einer Blume mit ihren Eigenschaften und ihrer Gatung.
\paragraph{Feature}
Ein Feature ist beschreibt die Eigenschaften eines Samples.
Im Falle des \ac{ids} hat jedes Sample vier Features.
\paragraph{Label}
Das Label drückt die Art des Samples aus, eine Zugehörigkeit zu einer Klasse.
Im Falle des \ac{ids} besitzt jedes Sample ein Label. 
Im ganzen Datensatz gibt es drei Labels, und zwar die drei Blumengattungen.

\subsection{Python}
\subsection{Python Bibliotheken}
\subsection{Deep Learning Frameworks}
\subsection{Keras}
\section{Das Neuron}
\subsection{Das Perceptron}
\subsection{Das ADALINE}
\subsection{Aktivierungsfunktionen}
\subsection{Fehlerfunktionen}
\subsection{Gradient Descent}
\subsection{Stochastic und Batch Gradient Descent}
\section{Künstliche Neuronale Netze}
\subsection{Definition}
\subsection{Arten des Deep Learning}
\subsection{Vom Neuron zum Deep Learning}
\subsection{Stochastic und Batch Gradient Descent}
\subsection{Backpropagation}
\subsection{Optimizer}
\subsection{Das Densenet}
\subsection{Das Convolutional Neural Network}
\subsection{Das Recurrent Neural Network}

%\begin{thebibliography}{00}
%\bibitem{b7} M. Young, The Technical Writer's Handbook. Mill Valley, CA: University Science, 1989.	
%\end{thebibliography}
\printbibliography
\end{document}
