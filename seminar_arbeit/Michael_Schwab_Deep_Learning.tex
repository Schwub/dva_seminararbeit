\documentclass[conference, german]{IEEEtran}
\IEEEoverridecommandlockouts
% The preceding line is only needed to identify funding in the first footnote. If that is unneeded, please comment it out.
\usepackage{cite}
\usepackage{amsmath,amssymb,amsfonts}
\usepackage{algorithmic}
\usepackage{graphicx}
\usepackage{textcomp}
\usepackage{xcolor}
\usepackage[utf8]{inputenc}
\usepackage[
type={CC},
modifier={by-nc-sa},
lang={german},
version={4.0},
hyphenation={raggedright}
]{doclicense}
\def\BibTeX{{\rm B\kern-.05em{\sc i\kern-.025em b}\kern-.08em
    T\kern-.1667em\lower.7ex\hbox{E}\kern-.125emX}}


\begin{document}

\title{Deep Learning\\
{\footnotesize DVA-Seminar 2018}
}
\author{\IEEEauthorblockN{Michael Schwab}
\IEEEauthorblockA{\textit{Fakultät für Informatik} \\
\textit{Hochschule für angewandte Wissenschaften Augsburg}\\
Augsburg, Deutschland \\
michael.schwab@hs-augsburg.de}
}

\maketitle
\renewcommand{\abstractname}{Zusammenfassung}



\begin{abstract}
\end{abstract}

%\begin{IEEEkeywords}
%\end{IEEEkeywords}	
\section*{Lizens}
\href{http://creativecommons.org/licenses/by-nc/4.0/}{\includegraphics{img/by-nc.png}}\\
{Deep Learning} von {Michael Schwab} ist lizenziert unter einer
\href{http://creativecommons.org/licenses/by-nc/4.0/}{Creative Commons
	Namensnennung-Nicht kommerziell 4.0 International Lizenz}.
\section{Einleitung} 
\section{Konventionen}
\section{Grundlagen}
\subsection{Datensätze}
\subsection{Python}
\subsection{Python Bibliotheken}
\subsection{Deep Learning Frameworks}
\subsection{Keras}
\section{Das Neuron}
\subsection{Das Perceptron}
\subsection{Das ADALINE}
\subsection{Aktivierungsfunktionen}
\subsection{Fehlerfunktionen  }
\subsection{Gradient Descent}
\section{Künstliche Neuronale Netze}
\subsection{Definition}
\subsection{Arten des Deep Learning}
\subsection{Vom Neuron zum Deep Learning}
\subsection{Stochastic und Batch Gradient Descent}
\subsection{Backpropagation}
\subsection{Optimizer}
\subsection{Das Densenet}
\subsection{Das Convolutional Neural Network}
\subsection{Das Recurrent Neural Network}
%\begin{thebibliography}{00}
%\bibitem{b7} M. Young, The Technical Writer's Handbook. Mill Valley, CA: University Science, 1989.	
%\end{thebibliography}
\end{document}
